\documentclass{article}
\usepackage[]{amsmath}
\let\product\Pi
\parindent=0pt\relax

\begin{document}
\title{Bayesian odds ratio of two multinomials}
\author{Luis Pedro Coelho}
\maketitle

\emph{The equations in this derivation have not been checked by automatic
software. Unfortunately, I cannot think of a way of checking them easily as I
have not found a piece of software which understood Dirichlet integrals in $k$
dimensions.}

\bigbreak

We are given two datasets $\mathcal{D}_1$ and~$\mathcal{D}_2$, which are
completely summarised as the output of a multinomial of size~$k$. The question
is: \emph{were these two datasets produced by a single multinomial or two
different multinomial?} We will call the single multinomial
hypothesis~$\mathcal{H}_1$ and the two multinomial hypothesis~$\mathcal{H}_2$.

If we call $\mathcal{D}$ the complete data, we are enquiring after:
\begin{equation}
\frac{P(\mathcal{H}_1|\mathcal{D})}{P(\mathcal{H}_2|\mathcal{D})}.
\label{eq:odds}
\end{equation}
If we assume a prior with the same value for both hypothesis, this is equivalent to
\begin{equation}
\frac{P(\mathcal{D}|\mathcal{H}_1)}{P(\mathcal{D}|\mathcal{H}_2)}.
\label{eq:odds-reversed}
\end{equation}
We will also assume that the two datasets are independent in the case of a single multinomial, i.e.,
\begin{equation}
\frac{P(\mathcal{D}|\mathcal{H}_1)}{P(\mathcal{D}|\mathcal{H}_2)} =
\frac{P(\mathcal{D}_1,\mathcal{D}_2|\mathcal{H}_1)}{P(\mathcal{D}_1,\mathcal{D}_2|\mathcal{H}_2)} =
\frac{P(\mathcal{D}_1|M)P(\mathcal{D}_2|M)}{P(\mathcal{D}_1,\mathcal{D}_2|M)},
\label{eq:p1p2-overp12}
\end{equation}
where $M$ is the multinomial model. We now consider what form
$P(\mathcal{D}_i|M)$ takes. The data is simply a vector of counts, which we
will call~$\vec x$. We need to integrate over all possible multinomials, which
are in turn defined by a vector~$\vec\theta$ in the simplex defined by $\sum_i
\theta_i = 1$. We use a Dirichlet prior, which needs a new parameter~$\vec\alpha$.
\begin{align}
P(\vec{c}|\alpha,M)
    &= \int P(\vec{c}|\vec{\theta}|M)P(\vec{\theta}|M) d\theta \\
    &= \frac{1}{D(\vec{\alpha})}{\int (\product_i \theta_i^{c_i})(\product_i \theta_i^{\alpha_i-1}) d\theta} \\
    &= \frac{1}{D(\vec{\alpha})}{\int \product_i \theta_i^{c_i + \alpha_i - 1} d\theta} \\
    &= \frac{D(\vec{c} + \vec{\alpha})}{D(\vec{\alpha})},
\end{align}
where $D$ is the Dirichlet normalizing constant:
\begin{equation}
D(\vec{\alpha}) = \frac{\product_i \Gamma(\alpha_i)}{\Gamma(\sum_i \alpha_i)},
\end{equation}
which is also sometimes called the multinomial Beta (but too many functions are
already called Beta).

Back to~(\ref{eq:p1p2-overp12}). For convenience, I will call the vector of
counts from $\mathcal{D}_1$, $\vec{x}$ and that from $\mathcal{D}_2$, $\vec{y}$.
(\ref{eq:p1p2-overp12}) expands to
\begin{equation}
\frac{D(\vec{x}+\alpha)}{D(\vec{\alpha})}\cdot
    \frac{D(\vec{y}+\alpha)}{D(\vec{\alpha})}\cdot
    \frac{D(\vec{\alpha})}{D(\vec{x}+\vec{y}+\alpha)} =
\frac{D(\vec{x}+\vec{\alpha})D(\vec{y}+\vec\alpha)}{D(\vec{x}+\vec{y}+\vec\alpha) D(\vec\alpha)}.
\end{equation}

With another another assumption, namely that $\alpha_i = 0$ (and $D(\vec\alpha)
= 1$, which implies an improper prior), we can get a nicer expression:
\begin{equation}
\frac{P(\mathcal{D}|\mathcal{H}_1)}{P(\mathcal{D}|\mathcal{H}_2)} = \frac{D(\vec{x})D(\vec{y})}{D(\vec{x}+\vec{y})}.
\label{eq:final-frac-a0}
\end{equation}

If we further define the Beta function $B$ as:
\begin{equation}
B(x,y) = \frac{\Gamma(x)\Gamma(y)}{\Gamma(x+y)},
\end{equation}
then (\ref{eq:final-frac-a0}) become:
\begin{equation}
\frac{P(\mathcal{D}|\mathcal{H}_1)}{P(\mathcal{D}|\mathcal{H}_2)} =
        \frac{D(\vec{x})D(\vec{y})}{D(\vec{x}+\vec{y})} =
        \product_i B(x_i, y_i).
\end{equation}
\end{document}

