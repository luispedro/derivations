\documentclass{article}
\usepackage[]{amsmath}
\usepackage{sagetex}

\begin{document}
\title{Integral of the product of two Gaussians}
\author{Luis Pedro Coelho}
\maketitle

\begin{sageblock}
x = var('x')
mu1 = var('mu1')
mu2 = var('mu2')
s1 = var('s1', latex_name=r'\sigma_1^2', domain='positive')
s2 = var('s2', latex_name=r'\sigma_2^2', domain='positive')
assume(s1 > 0)
assume(s2 > 0)

def Z(s):
    return sqrt(2*pi*s)

def N(x, m, s):
    return 1./Z(s) * exp(- (x-m)^2 /(2*s))
\end{sageblock}

\begin{sageblock}
product = N(x, mu1, s1) * N(x,mu2,s2)
\end{sageblock}

We want to be able to compute:
\begin{equation}
\int_{-\infty}^{\infty} \sage{latex(product)} dx,
\end{equation}
or, in sage:
\begin{sageblock}
Nint = integral(product,x,-infinity,infinity)
\end{sageblock}

I assert that this is equal to:
\begin{sageblock}
s12 = s1*s2/(s1+s2)
Ndirect = Z(s12)/(Z(s1)*Z(s2)) * exp(- (mu1-mu2)^2/2/(s1+s2))
\end{sageblock}
where
\begin{align}
\sigma^2_{12} &= \sage{latex(s12)}\\
N &= \sage{latex(Ndirect)}
\end{align}

Let us check the ratio again
\begin{sageblock}
ratio = (Nint/Ndirect)
ratio = ratio.simplify_full()
\end{sageblock}
The ratio is $\sage{latex(ratio)}$.

We can also write the function above as
\begin{sageblock}
Ngaussian1 = Z(s12)*Z(s1+s2)/(Z(s1)*Z(s2)) *N(mu1,mu2,s1+s2)
\end{sageblock}
The products of all the $Z$s is going to simplify to 1:
\begin{sageblock}
Zs = Z(s12)*Z(s1+s2)/(Z(s1)*Z(s2))
Zs = Zs.simplify_full()
\end{sageblock}
Results in $\sage{latex(Zs)}$.

So, we get our final result:
\begin{sageblock}
Ngaussian = N(mu1,mu2,s1+s2)
ratio = (Nint/Ngaussian)
ratio = ratio.simplify_full()
\end{sageblock}
The ratio is, again, $\sage{latex(ratio)}$.

Therefore:
\begin{equation}
\int N(x|\mu_1, \sigma_1^2) N(x|\mu_2, \sigma_2^2) dx =
    N(\mu_1|\mu_2, \sigma_1^2 + \sigma_2^2) =
    N(\mu_2|\mu_1, \sigma_1^2 + \sigma_2^2).
\end{equation}

\end{document}

