\documentclass{article}
\usepackage[]{amsmath}
\usepackage{sagetex}

\begin{document}
\title{Product of two Gaussian probabilistic density functions}
\author{Luis Pedro Coelho}
\maketitle

Let's start by defining a few variables:
\begin{sageblock}
x = var('x')
mu1 = var('mu1')
mu2 = var('mu2')
\end{sageblock}
$x$ will be our variable, $\mu_1$ and $\mu_2$ the means.

\begin{sageblock}
s1 = var('s1', latex_name=r'\sigma_1^2', domain='positive')
s2 = var('s2', latex_name=r'\sigma_2^2', domain='positive')
N1, N2 = var('N1 N2')
\end{sageblock}
We define $\sigma^2$ as positive.

\begin{sageblock}
Z(s) = sqrt(2*pi*s)
N(x,m,s) = 1./Z(s) * exp(- (x-m)^2 /(2*s))

N1 = N(x, mu1, s1)
N2 = N(x, mu2, s2)
\end{sageblock}

\texttt{N1} is the first normal:
\begin{equation}
\sage{latex(N1)},
\end{equation}
which is not in canonical form because of sage simplification, but we can
recognise the normal distribution.

\begin{sageblock}
product = N1 * N2
\end{sageblock}


The product as a Gaussian is readily obtained by algebraic manipulation:
\begin{sageblock}
m12 = (mu1*s2+mu2*s1)/(s1+s2)
s12 = s1*s2/(s1+s2)
newnormal = N(x,m12,s12)
\end{sageblock}

We have that
\begin{align}
\mu_{12} &= \sage{latex(m12)} \\
\sigma_{12}^2 &= \sage{latex(s12)}
\end{align}

However, to obtain the full product, we need to add several normalization factors:
\begin{sageblock}
direct = 1./Z(s1)*1./Z(s2)*\
        exp(- (mu1-mu2)^2/2/(s1+s2))*Z(s12)*newnormal
\end{sageblock}
Which is
\begin{equation}
\sage{latex(direct)}.
\end{equation}

Let's finally check that this is correct, by checking the value of the ratio:
\begin{sageblock}
final = (product/direct).full_simplify()
\end{sageblock}

And the output is $\sage{final}$!

\end{document}

